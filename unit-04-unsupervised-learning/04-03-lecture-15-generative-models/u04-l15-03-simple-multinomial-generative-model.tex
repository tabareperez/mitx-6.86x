% Created 2020-05-06 mié 13:58
% Intended LaTeX compiler: pdflatex
\documentclass[a4paper, 12pt]{article}
\usepackage[utf8]{inputenc}
\usepackage[T1]{fontenc}
\usepackage{graphicx}
\usepackage{grffile}
\usepackage{longtable}
\usepackage{wrapfig}
\usepackage{rotating}
\usepackage[normalem]{ulem}
\usepackage{amsmath}
\usepackage{textcomp}
\usepackage{amssymb}
\usepackage{capt-of}
\usepackage{hyperref}
\usepackage{float, amsfonts, commath, mathtools, proba}
\author{Tabaré Pérez}
\date{\today}
\title{Lecture 15 - 3: Simple Multinomial Generative model}
\hypersetup{
 pdfauthor={Tabaré Pérez},
 pdftitle={Lecture 15 - 3: Simple Multinomial Generative model},
 pdfkeywords={},
 pdfsubject={},
 pdfcreator={Emacs 26.3 (Org mode 9.3.6)}, 
 pdflang={English}}
\begin{document}

\maketitle
So whenever we are talking about multinomials, one commonly example used is
talking about documents.

So we will talk about multinomials, and in your head, you can think about texts,
documents and what our models, our multinomial models will do, they will
generate documents.

Now, whenever we were talking about supervised learning, you remember we
discussed how to translate, we thought about the documents, and we translated
them into a vector, into bag of words. So there was some particular mechanism
that, given a document, we generated a vector of fixed lengths.

Here, we are going to be thinking about it in a different way. We're going to
think that our model is going to be generating documents. And some documents,
which are good for this model, will have very high probability, and other
documents may have low probability.

So let me first show you some notation so that we can ground this discussion in
the type of language that we will need to use to discuss our questions of
estimation and predictions. And before I go, I want to say that, of course,
you're thinking:

\begin{itemize}
\item How probabilistic model can be generating documents?
\item Is it going to write poems and essays?
\end{itemize}

So when I'm talking about generate, it's not like generate an essay.

We'll have a very simplistic definition of generate. Specifically, what we would
assume that this model have fixed vocabulary, first of all.

We, as humans, also have fixed vocabulary. This is capital \(\mathcal{W}\) and then
these models would generate one word at a time. So you have this whole bag of
possible words, you select one word, put it there, and then again, you go to the
same bag, and you select another word, and all the words that we are selecting
are independent of each other.

So it's a very simplistic thing, because clearly, we're not going to get
beautiful sentences.

But this is our first model that we will use. So again, the words all will come
from the same vocabulary. We draw them one at a time and fully independently.

So what kind of parameters do we need to have to talk about this model? So the
first thing in multinomials, we need to decide how likely it is to generate
certain words.

Because depending on the model, certain words will be more likely or less
likely. So one of the parameters here would be something which is the likelihood
of generating the word \(w\), given parameterization of the model:

\begin{equation}
\prob(w|\theta) = \theta_w
\end{equation}

So \(\theta\) means parameters of the model. So in this case, just for ease of
writing, I would write it as \(\theta_w\). So \(\theta_w\) captures what is the
likelihood of selecting, generating certain words given all the possibilities.

So what kind of constraints do we need to have on the \(\theta_w\) to make sure
that we have a valid probability distribution?

We need to make sure, because this is probabilities:

\begin{enumerate}
\item \(\theta_w \geq 0\)
\item \(\sum_{w \in \mathcal{W}} \theta_w = 1\)
\end{enumerate}

So this is our \textbf{MULTINOMIAL}.

And if you need another example and the text sounds to you really weird, you can
just see about dice:

So you're throwing dice, and your dice doesn't have all the equal sides. Some
sides are more likely than others.

And that's exactly what  \(\theta_w\) would record: the likelihood of a
particular word or number.
\end{document}