% Created 2020-05-06 mié 14:00
% Intended LaTeX compiler: pdflatex
\documentclass[a4paper, 12pt]{article}
\usepackage[utf8]{inputenc}
\usepackage[T1]{fontenc}
\usepackage{graphicx}
\usepackage{grffile}
\usepackage{longtable}
\usepackage{wrapfig}
\usepackage{rotating}
\usepackage[normalem]{ulem}
\usepackage{amsmath}
\usepackage{textcomp}
\usepackage{amssymb}
\usepackage{capt-of}
\usepackage{hyperref}
\usepackage{float, amsfonts, commath, mathtools, proba}
\author{Tabaré Pérez}
\date{\today}
\title{Lecture 15 - 5: Maximum Likelihood Estimate}
\hypersetup{
 pdfauthor={Tabaré Pérez},
 pdftitle={Lecture 15 - 5: Maximum Likelihood Estimate},
 pdfkeywords={},
 pdfsubject={},
 pdfcreator={Emacs 26.3 (Org mode 9.3.6)}, 
 pdflang={English}}
\begin{document}

\maketitle
So now we have our parameters. We have ways to compute the likelihood of the
documents based on these parameters.

And the question is, how can we utilize our training data to find the best
parameters.

So again, we are coming back to supervised scenario that you've seen in the case
of classifiers.

We will provide our model with, let's say, positive or negative examples, and
the model needs to find the best parameters that fit the data. Here we are going
to go back to our favorite principles of maximum likelihood, and we will make
the assumption that the best parameters are the parameters which give the
highest likelihood to our data.

So you have the training data and you're trying to find parameters for which the
probability of these training data will be the highest.

So in other words, what we are trying to do is to find these \(\theta\)s which
maximize this expression:

\begin{equation}
\max_{\theta} \prob (D|\theta)=\max_{\theta}\prod_{w \in \mathcal{W}}\theta_{w}^{\text{count}(w)}
\end{equation}

And again, assume that this is your training document \(D\), over all the
\(\theta\)s.

So it turns out, and you will see it both in the lecture and in the exercises,
that it's easier for us to this maximization to find the best \(\theta\)s and
instead of working with this expression directly, we would actually maximize the
log, and we can do it. So what we will need to do is to take the log of this
expression:

\begin{equation}
\log\prod_{w \in \mathcal{W}} \theta_{w}^{\text{count}(w)}=\sum_{w \in \mathcal{W}} \text{count}(w) \cdot \log\theta_{w}
\end{equation}

You remember that log of the products is actually sum of logs.

So now we are trying to find \(\theta\)s that are going to maximize the value of
this expression. What I will do now, I will stick to the example that I had
before where I really looked at the vocabulary of only two symbols.

We will first solve this for the case where the alphabet has just two symbols,
and then discussed how we can solve it for general vocabulary. And I know that
we all love cats and dogs, but for me it will be easier to write \(\mathcal{W}\)
with just two words, exactly like in the example above, but this time I'm going
to be using just the words \(0\) and \(1\), because it's shorter to write it.

\begin{equation}
\mathcal{W}=\{0, 1\}
\end{equation}

In this particular case, what's so special above this case?

Because in this case, we actually just need it to have a single \(\theta\).

So we can write that :

\begin{equation}
\theta_0=\theta, \theta_1=1-\theta
\end{equation}

We need a \(\theta\) for zero, the likelihood of generating the word zero. If we want
to know the \(\theta_1\), we can just do \(1\) and subtract \(\theta\).

So I would just have one parameter, and then I'm going to rewrite this formula
with the single parameter.

\begin{equation}
\sum_{w \in \mathcal{W}} \text{count}(w) \cdot \log(\theta_{w})=\text{count}(0) \cdot \log(\theta)+\text{count}(1) \cdot \log(1-\theta)
\end{equation}

So my next step is actually to find derivative of this expression with respect
to \(\theta\) and what we will get in this case is:

\begin{equation}
\frac{\partial}{\partial \theta} [\text{count}(0) \cdot \log(\theta)+\text{count}(1)\cdot \log(1-\theta)]=\frac{\text{count}(0)}{\theta}-\frac{\text{count}(1)}{(1-\theta)}
\end{equation}

So this is our new expression, and we need to make it equal to \(0\) to find our
desired \(\theta\)s. So in this case, we can just do this computation:

\begin{equation}
(1-\theta)\cdot \text{count}(0) - \theta \cdot \text{count}(1) = 0
\end{equation}

So if you go and do the algebra, which is pretty straightforward in this case
what you would get:

\begin{equation}
\hat{\theta}=\frac{\text{count}(0)}{\text{count}(0)+\text{count}(1)}
\end{equation}

You see I use this estimate \(\hat{\theta}\), special notation, would be actually
equal to something that sounds like very logical.

This is kind of an obvious thing, so it looks weird.

You say, for example, I have a document which have 20 zeros and 10 ones. So what
is the likelihood of observing \(0\) here? Would be 20 divided by 30, and that's
exactly what this formula says.

But what we've demonstrated here is that, by properly defined our maximum
likelihood criteria and making the estimation, we're actually getting the
parameters that we are expected to get.

And in some cases, for some model, you can kind of intuitively say what they
should look like. In other cases not, but this mechanism always works.
\end{document}