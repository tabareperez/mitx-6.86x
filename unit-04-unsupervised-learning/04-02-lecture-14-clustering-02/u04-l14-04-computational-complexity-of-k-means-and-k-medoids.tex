% Created 2020-05-06 mié 13:55
% Intended LaTeX compiler: pdflatex
\documentclass[a4paper, 12pt]{article}
\usepackage[utf8]{inputenc}
\usepackage[T1]{fontenc}
\usepackage{graphicx}
\usepackage{grffile}
\usepackage{longtable}
\usepackage{wrapfig}
\usepackage{rotating}
\usepackage[normalem]{ulem}
\usepackage{amsmath}
\usepackage{textcomp}
\usepackage{amssymb}
\usepackage{capt-of}
\usepackage{hyperref}
\usepackage{float, amsfonts, commath, mathtools}
\author{Tabaré Pérez}
\date{\today}
\title{Lecture 14 - 4: Computational Complexity of K-Means and K-Medoids}
\hypersetup{
 pdfauthor={Tabaré Pérez},
 pdftitle={Lecture 14 - 4: Computational Complexity of K-Means and K-Medoids},
 pdfkeywords={},
 pdfsubject={},
 pdfcreator={Emacs 26.3 (Org mode 9.3.6)}, 
 pdflang={English}}
\begin{document}

\maketitle
So what are the other interesting differences between these two algorithms?

One difference that you can see when you actually made this computation is
that this computation clearly seems to us more expensive than the computations that
we are doing in K-means.

So if we are to use capital \(\mathcal{O}\) notation, which talks to us about the order
of growth, let's look at this algorithm and compare them in terms of their time
complexity.

Now, note that we don't know how many steps each of these algorithms will take
to find as a partitioning, but what we will look at is the cost of one iteration
of the algorithm. And then it will take many, many iterations until convergence.
So let's start with the k-means algorithm:

\begin{enumerate}
\item Ramdomly initialize \(z^{(1)} \ldots z^{(K)}\)
\item Iterate unitl no change in cost:
\begin{itemize}
\item 2a. for \(i=1 \ldots n\):

\(C_j = \{i|\) s.t. \(z^{(j)}\) is closest to \(x^{(i)}\}\)

\item 2b. for \(j=1 \ldots K\):

\(z^{(j)} = \frac{\sum_{i \in C_j} x^{(i)}}{\abs{C_j}}\)
\end{itemize}
\end{enumerate}

So in this particular case, again, we're talking about the order of growth,
which means that we are going to look at the asymptotic growth and eliminate all
the constants. So we can see here that, whenever we are doing both of these
computations, they will cost us \(n\), because we are doing it for every point,
\(K\) the number of clusters, and may be multiplied by dimensionality \(d\),
which is the size of the cluster:

\begin{equation}
\mathcal{O}(nKd)
\end{equation}

So let me just explain, what do I mean?

So for instance, for stage \textbf{2a}, you need to go for each point and find for it
the closest representative.

So this, the order of computation will be \(n\) multiplied by \(K\), because you
have \(n\) points and you have \(K\) representatives.

And the reason we are multiplying by \(d\), is that becuase our \(x\)’s may be
very high dimensional vectors and if we want to account for it, we can add it to
our complexity.

So this will be complexity for the k-means algorithm. And you can see that the
similar complexity is achieved by the state 2b.

But again, because we're talking about the asymptotic growth, whenever we sum
them up, we're still staying within the same order of complexity.

Now, let's look at the complexity of the k-medoids algorithm:

\begin{enumerate}
\item Ramdomly initialize \(\{z^{(1)} \ldots z^{(K)}\} \subseteq \{x^{(1)} \ldots x^{(n)}\}\)
\item Iterate until there is no change in cost:
\begin{itemize}
\item 2a. for \(i=1 \ldots n\):

\(C_j = \{i|\) s.t. \(z^{(j)}\) is closest to \(x^{(i)}\}\)

\item 2b. for \(j=1 \ldots K\):

\(z^{(j)} \in \{x^{(1)} \ldots x^{(n)}\} | \sum_{i \in C_j} \text{distance}(x^{(i)}, z^{(j)})\) is minimal.
\end{itemize}
\end{enumerate}


So in this state, this step of the algorithm, \textbf{2a} costs exactly the same.

However, here we are going to be paying much more, because now we need to
compute the distances between every pair of points.

And if we're using kind of straightforward implementation, the order that we are
going to be considering is:

\begin{equation}
\mathcal{O}(n^{2}Kd)
\end{equation}

I'm talking about the most naïve implementation and there are ways to
implement it in better ways.

But roughly speaking, however you're not going to implement it, it's intuitively
clear to us that this is more expensive computation and for somebody may be a
decision to select one algorithm versus another.

But at this point, we already have seen two clustering algorithms. There are
hundreds of them available for your use and the reason I wanted to show them,
even though they're close to each other, is that they have different strengths
and weaknesses.

And whenever you are selecting clustering algorithm which fits for your
application, you may want to think about different consideration when you're
finding the best clustering algorithm for your needs.
\end{document}